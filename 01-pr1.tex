\section{Příklad 1}
% Jako parametr zadejte skupinu (A-H)
\prvniZadani{C}
\subsection{Výpočet odporu $R_{ekv}$}
\begin{figure}[H]
  Zjednodušení sériově zapojených zdrojů: $ U = U_1 + U_2$

  \begin{circuitikz}
    \draw
    (0,0) to[dcvsource, v^<=U](0,4)
    (0,4) --                  (1,4)
    (1,4) node[circ]{}        (1,4)
    (1,4) --                  (1,2)
    (1,2) to[R, l^=$R_2$]     (4,2)
    (1,4) to[R, l^=$R_1$]     (4,4)
    (4,4) to[R, l^=$R_3$]     (4,2)
    (4,2) node[circ]{}        (4,2)
    (4,2) to[R, l^=$R_6$]     (7,2)
    (4,4) node[circ]{}        (4,4)
    (7,4) node[circ]{}        (7,4)
    (4,4) --                  (4,5)
    (7,4) --                  (7,5)
    (4,4) to[R, l^=$R_{5}$]   (7,4)
    (4,5) to[R, l^=$R_{4}$]   (7,5)
    (7,4) to[R, l^=$R_7$]     (7,2)
    (7,2) node[circ]{}        (7,2)
    (7,2) --                  (7,0)
    (0,0) --                  (1,0)
    (1,0) to[R, l^=$R_8$]     (4,0)
    (4,0) --                  (7,0)
    ;
  \end{circuitikz}
\end{figure}

\begin{figure}[H]
  Zjednodušení paralelně zapojených rezistorů: $ R_{45} = \frac{R_4 \cdot R_5}{R_4 + R_5}$

  \begin{circuitikz}
    \draw
    (0,0) to[dcvsource, v^<=U](0,4)
    (0,4) --                  (1,4)
    (1,4) node[circ]{}        (1,4)
    (1,4) --                  (1,2)
    (1,2) to[R, l^=$R_2$]     (4,2)
    (1,4) to[R, l^=$R_1$]     (4,4)
    (4,4) to[R, l^=$R_3$]     (4,2)
    (4,2) node[circ]{}        (4,2)
    (4,2) to[R, l^=$R_6$]     (7,2)
    (4,4) node[circ]{}        (4,4)
    (4,4) to[R, l^=$R_{45}$]  (7,4)
    (7,4) to[R, l^=$R_7$]     (7,2)
    (7,2) node[circ]{}        (7,2)
    (7,2) --                  (7,0)
    (0,0) --                  (1,0)
    (1,0) to[R, l^=$R_8$]     (4,0)
    (4,0) --                  (7,0)
    ;
  \end{circuitikz}
\end{figure}

\begin{figure}[H]
  Zjednodušení sériově zapojených rezistorů: $ R_{457} = R_{45} + R_7$

  \begin{circuitikz}
    \draw
    (0,0) to[dcvsource, v^<=U](0,4)
    (0,4) --                  (1,4)
    (1,4) node[circ]{}        (1,4)
    (1,4) --                  (1,2)
    (1,2) to[R, l^=$R_2$]     (4,2)
    (1,4) to[R, l^=$R_1$]     (4,4)
    (4,4) to[R, l^=$R_3$]     (4,2)
    (4,2) node[circ]{}        (4,2)
    (4,2) to[R, l^=$R_6$]     (7,2)
    (4,4) node[circ]{}        (4,4)
    (4,4) to[R, l^=$R_{457}$] (7,4)
    (7,4) --                  (7,2)
    (7,2) node[circ]{}        (7,2)
    (7,2) --                  (7,0)
    (0,0) --                  (1,0)
    (1,0) to[R, l^=$R_8$]     (4,0)
    (4,0) --                  (7,0)
    ;
  \end{circuitikz}
\end{figure}

\begin{figure}[H]
  Transfigurace trojúhelník na hvězdu

  \begin{circuitikz}
    \draw
    (0,0) to[dcvsource, v^<=U](0,3)
    (0,3) to[R, l^=$R_A$]     (2,3)
    (2,3) node[circ]{}        (2,3)
    (2,3) to[R, l^=$R_B$]     (4,4)
    (2,3) to[R, l^=$R_C$]     (4,2)
    (4,2) to[R, l^=$R_6$]     (7,2)
    (4,4) to[R, l^=$R_{457}$] (7,4)
    (7,4) --                  (7,2)
    (7,2) node[circ]{}        (7,2)
    (7,2) --                  (7,0)
    (0,0) --                  (1,0)
    (1,0) to[R, l^=$R_8$]     (4,0)
    (4,0) --                  (7,0)
    ;
  \end{circuitikz}

  \begin{equation*}
    \begin{aligned}
      R_A & = \frac{R_1 \cdot R_2}{R_1 + R_2 + R_3} \\
      R_B & = \frac{R_1 \cdot R_3}{R_1 + R_2 + R_3} \\
      R_C & = \frac{R_2 \cdot R_3}{R_1 + R_2 + R_3}
    \end{aligned}
  \end{equation*}
\end{figure}

\begin{figure}[H]
  Zjednodušení sériově zapojených rezistorů: $ R_{B457} = R_B + R_{457}, R_{C6} = R_C + R_6$

  \begin{circuitikz}
    \draw
    (0,0) to[dcvsource, v^<=U](0,3)
    (0,3) to[R, l^=$R_A$]     (3,3)
    (3,3) node[circ]{}        (3,3)
    (3,2) to[R, l^=$R_{C6}$]  (6,2)
    (3,4) to[R, l^=$R_{B457}$](6,4)
    (3,2) --                  (3,4)
    (6,4) --                  (6,2)
    (6,2) node[circ]{}        (6,2)
    (6,2) --                  (6,0)
    (0,0) to[R, l^=$R_8$]     (3,0)
    (3,0) --                  (6,0)
    ;
  \end{circuitikz}
\end{figure}

\begin{figure}[H]
  Zjednodušení paralelně zapojených rezistorů: $ R_{BC4567} = \frac{R_{B457} \cdot R_{C6}}{R_{B457} + R_{C6}}$

  \begin{circuitikz}
    \draw
    (0,0) to[dcvsource, v^<=U]  (0,3)
    (0,3) to[R, l^=$R_A$]       (3,3)
    (3,3) to[R, l^=$R_{BC4567}$](6,3)
    (6,3) --                    (6,2)
    (6,2) --                    (6,0)
    (0,0) to[R, l^=$R_8$]       (3,0)
    (3,0) --                    (6,0)
    ;
  \end{circuitikz}
\end{figure}

\begin{figure}[H]
  Zjednodušení sériově zapojených rezistorů: $ R_{ekv} = R_A + R_{BC4567} + R_8 $

  \begin{circuitikz}
    \draw
    (0,0) to[dcvsource, v^<=U]  (0,3)
    (0,3) --                    (3,3)
    (3,3) to[R, l^=$R_{ekv}$]   (3,0)
    (0,0) --                    (3,0)
    ;
  \end{circuitikz}
\end{figure}

\subsection{Výpočet $U_{R7}$ a $I_{R7}$}
\begin{figure}[H]
  Výpočet celkového proudu v obvodu
  $$ I = \frac{U}{R_{ekv}} $$
  Zpětné dopočítání proudu a napětí na $R_7$

  \begin{circuitikz}
    \draw
    (0,0) to[dcvsource, v^<=U](0,3)
    (0,3) to[R, l^=$R_A$]     (2,3)
    (2,3) node[circ]{}        (2,3)
    (2,3) to[R, l^=$R_B$]     (4,4)
    (2,3) to[R, l^=$R_C$]     (4,2)
    (4,2) to[R, l^=$R_6$]     (7,2)
    (4,4) to[R, l^=$R_{457}$] (7,4)
    (7,4) --                  (7,2)
    (7,2) node[circ]{}        (7,2)
    (7,2) --                  (7,0)
    (0,0) --                  (1,0)
    (1,0) to[R, l^=$R_8$]     (4,0)
    (4,0) --                  (7,0)
    ;
  \end{circuitikz}

  Nejprve vypočítáme napětí na $U_{BC4567}$. Poté proud v horní větvi, využíváme toho, že $U_{B457} = U_{BC4567}$, tento proud se rovná $I_{R7}$. Jako poslední krok vypočítáme podle ohmova zákona $U_{R7}$.
  \begin{equation*}
    \begin{aligned}
      U_{BC4567} & = R_{BC4567} I                \\
      I_{B457}   & = \frac{U_{BC4567}}{R_{B457}} \\
      I_{R7}     & = I_{B457}                    \\
      U_{R7}     & = R_7 I_{R7}                  \\
    \end{aligned}
  \end{equation*}


\end{figure}

\subsection{Dosazení}
\begin{figure}[H]

  \begin{equation*}
    \begin{aligned}
      U          & = U_1 + U_2 = 100 + 80 = 180V                                                                                                                          \\
      R_{45}     & = \frac{R_4 \cdot R_5}{R_4 + R_5} = \frac{220 \cdot 220}{220 + 220} = 110 \ohm                                                                         \\
      R_{457}    & = R_{45} + R_7 = 110 + 260 = 370 \ohm                                                                                                                  \\
      R_A        & = \frac{R_1 \cdot R_2}{R_1 + R_2 + R_3} = \frac{450 \cdot 810}{450 + 810 + 190} = \frac{7290}{29} \ohm                                                 \\
      R_B        & = \frac{R_1 \cdot R_3}{R_1 + R_2 + R_3} = \frac{450 \cdot 190}{450 + 810 + 190} = \frac{1710}{29} \ohm                                                 \\
      R_C        & = \frac{R_2 \cdot R_3}{R_1 + R_2 + R_3} = \frac{810 \cdot 190}{450 + 810 + 190} = \frac{3078}{29} \ohm                                                 \\
      R_{B457}   & = R_B + R_{457} = \frac{1710}{29} + 370 = \frac{12440}{29} \ohm                                                                                        \\
      R_{C6}     & = R_C + R_6 = \frac{3078}{29} + 720 = \frac{23958}{29} \ohm                                                                                            \\
      R_{BC4567} & = \frac{R_{B457} \cdot R_{C6}}{R_{B457} + R_{C6}} = \frac{\frac{12440}{29} \cdot \frac{23958}{29}}{\frac{12440}{29} + \frac{23958}{29}} = 282.354 \ohm \\
      R_{ekv}    & = \R_A + R_{BC4567} + R_8 = \frac{7290}{29} + 282.354 + 180 = 713.7343 \ohm                                                                            \\
      I          & = \frac{U}{R_{ekv}} = \frac{180}{713.7343} = 0.2522 A                                                                                                  \\
      U_{BC4567} & = R_{BC4567} I = 282.354 \cdot 0.2522 = 71.2084 V                                                                                                      \\
      I_{R7}     & = \frac{U_{BC4567}}{R_{B457}} = I_{B457} = \frac{71.2084}{\frac{12440}{29}} = \underline{\underline{1.66 \cdot 10^{-1} A}}                             \\
      U_{R7}     & = R_7 I_{R7} = 260 * 1.66 \cdot 10^{-1} = \underline{\underline{43.1601 V}}
    \end{aligned}
  \end{equation*}
\end{figure}
