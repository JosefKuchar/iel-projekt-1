\section{Příklad 3}
% Jako parametr zadejte skupinu (A-H)
\tretiZadani{G}

\subsection{Převedení zdroje a zvolení proudů}
\begin{figure}[H]
  Schéma po převedení napěťového zdroje na proudový s vyznačenými zvolenými proudy.
  \begin{circuitikz}
    \draw
    (0,0) to[ioosource, i^=$I_1$]  (0,3)
    (0,3) -* (2,3)
    (2,0) to[R, l^=$G_1$, i<=$I_{G1}$]     (2,3)
    (2,3) to[R, l^=$G_4$, i<=$I_{G4}$]     (5,3)
    (2,4) to[R, l^=$G_5$, i<=$I_{G5}$]     (5,4)
    (2,5) to[ioosource, i<=$I_Z$]  (5,5)
    (2,3) -- (2,5)
    (5,3) -- (5,5)
    (5,3) to[R, l^=$G_3$, i=$I_{G3}$]     (5,0)
    (0,0) -- (2,0)
    (2,0) to[R, l^=$G_2$, i<=$I_{G2}$]     (5,0)
    (5,0) -- (7,0)
    (7,0) to[ioosource, i^=$I_2$]  (7,3)
    (5,3) -- (7,3)
    (2,0) node[circ]{}        (2,0)
    (5,0) node[circ]{}        (5,0)
    (2,3) node[circ]{}        (2,3)
    (2,4) node[circ]{}        (2,4)
    (5,3) node[circ]{}        (5,3)
    (5,4) node[circ]{}        (5,4)
    {[anchor=south east] (2,3) node{A}}
      {[anchor=south west] (5,3) node{B}}
      {[anchor=north west] (5,0) node{C}}
    ;
  \end{circuitikz}
\end{figure}

\subsection{Vyjádření vodivostí}
\begin{figure}[H]
  Nejprve převedeme všechny odpory na vodivosti

  $$ G_1 = \frac{1}{R_1} $$
  $$ G_2 = \frac{1}{R_2} $$
  $$ G_3 = \frac{1}{R_3} $$
  $$ G_4 = \frac{1}{R_4} $$
  $$ G_5 = \frac{1}{R_5} $$
\end{figure}

\subsection{Výpočet napětí}
\begin{figure}[H]
  Sestavené rovnice pro uzly použítím I. Kirchhoffova zákona
  \begin{equation*}
    \begin{aligned}
      I_1 - I_{G1} + I_{G4} + I_{G5} + I_Z & = 0 \\
      I_2 - I_{G3} - I_{G4} - I_{G5} - I_Z & = 0 \\
      I_{G3} - I_{G2} - I_2                & = 0
    \end{aligned}
  \end{equation*}


  Vyjádření neznámých proudů pomocí napětí a vodivostí

  \begin{equation*}
    \begin{aligned}
      -G_1 U_A + G_4 (U_B - U_A) + G_5 (U_B - U_A)         & = -I_1 - I_Z \\
      -G_3 (U_B - U_C) - G_4 (U_B - U_A) - G_5 (U_B - U_A) & = I_Z - I_2  \\
      G_3 (U_B - U_C) - G_2 U_C                            & = I_2
    \end{aligned}
  \end{equation*}


  Po vytknutní napětí dostaváme

  \begin{equation*}
    \begin{aligned}
      U_A (-G_1 - G_4 - G_5) + U_B (G_4 + G_5)            & = -I_1 - I_Z \\
      U_A (G_4 + G_5) + U_B (-G_3 - G_4 -G_5) + U_C (G_3) & = I_Z - I_2  \\
      U_B (G_3) + U_C (-G_3 - G_2)                        & = I_2
    \end{aligned}
  \end{equation*}

  Následně soustavu rovnic převedeme do matice

  $$
    \begin{pmatrix}
      -G_1-G_4-G_5 & G_4+G_5      & 0        \\
      G_4+G_5      & -G_3-G_4-G_5 & G_3      \\
      0            & G_3          & -G_3-G_2
    \end{pmatrix}
    \cdot
    \begin{pmatrix}
      U_A \\
      U_B \\
      U_C
    \end{pmatrix}
    =
    \begin{pmatrix}
      -I_1-I_Z \\
      I_Z-I_2  \\
      I_2
    \end{pmatrix}
  $$
\end{figure}

\subsection{Výpočet požadovaného napětí a proudu}
\begin{figure}[H]
  Po vypočítání matice už snadno vypočítáme napětí $U_{R3}$ a proud $I_{R3}$

  $$ U_{R3} = U_B - U_C $$
  $$ I_{R3} = U_{R3} G_{3} $$
\end{figure}

\subsection{Dosazení}
\begin{figure}[H]

  $$ G_1 = \frac{1}{R_1} = \frac{1}{46} S $$
  $$ G_2 = \frac{1}{R_2} = \frac{1}{41} S $$
  $$ G_3 = \frac{1}{R_2} = \frac{1}{53} S$$
  $$ G_4 = \frac{1}{R_4} = \frac{1}{33} S$$
  $$ G_5 = \frac{1}{R_5} = \frac{1}{29} S$$

  $$
    \begin{pmatrix}
      -0.086525 & 0.064786  & 0         \\
      0.064786  & -0.083654 & 0.018868  \\
      0         & 0.018868  & -0.043258
    \end{pmatrix}
    \cdot
    \begin{pmatrix}
      U_A \\
      U_B \\
      U_C
    \end{pmatrix}
    =
    \begin{pmatrix}
      -6.1672 \\
      5.0672  \\
      0.4500
    \end{pmatrix}
  $$

  $$
    \begin{pmatrix}
      U_A \\
      U_B \\
      U_C
    \end{pmatrix}
    =
    \begin{pmatrix}
      53.3124  \\
      -23.9928 \\
      -20.8676
    \end{pmatrix}
  $$

  \begin{equation*}
    \begin{aligned}
      U_{R3} & = U_B - U_C = -23.9928 - (-20.8676) = \underline{\underline{-3.1252 V}}                       \\
      I_{R3} & = U_{R3} G_{3} = -3.1252 \cdot \frac{1}{53} = \underline{\underline{-5.8965 \cdot 10^{-2} A}}
    \end{aligned}
  \end{equation*}

\end{figure}
