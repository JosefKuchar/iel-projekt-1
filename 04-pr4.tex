\section{Příklad 4}
% Jako parametr zadejte skupinu (A-H)
\ctvrtyZadani{C}

\subsection{Zvolení smyček}
\begin{figure}[H]
  \begin{circuitikz}
    \ctikzset{cute inductors}
    \draw
    (0,0) -- (0,6)
    (0,0) to[C, l=$C_1$] (4,0)
    (4,0) to[R, l=$R_2$] (8,0)
    (0,3) to[R, l=$R_2$] (4,3)
    (4,3) to[C, l=$C_2$] (8,3)
    (0,6) to[L, l=$L_1$] (4,6)
    (8,3) to[L, l=$L_2$, i=$I_{L2}$, v=$U_{L2}$] (8,0)
    (8,6) -- (8,3)
    (8,6) to[vsourcesin, v_<=$U_1$] (4,6)
    (4,0) to[vsourcesin, v<=$U_2$] (4,3)
    (4,0) node[circ]{} (4,0)
    (4,3) node[circ]{} (4,3)
    (8,3) node[circ]{} (8,3)
    (0,3) node[circ]{} (0,3)
    ;
    \draw[-stealth] (4,4.5)node{$I_A$}  ++(-60:0.5) arc (-60:170:0.5);
    \draw[-stealth] (2,1.5)node{$I_B$}  ++(-60:0.5) arc (-60:170:0.5);
    \draw[stealth-] (6,1.5)node{$I_C$}  ++(-60:0.5) arc (-60:170:0.5);
  \end{circuitikz}
\end{figure}

\subsection{Vyjádření impedancí}
\begin{figure}[H]
  Nejprve vyjádříme úhlovou frekvenci
  $$ \omega = 2 \pi f $$

  Poté vyjádříme impedance kondenzátorů a cívek

  \begin{equation*}
    \begin{aligned}
      Z_{L1} & = j L_1 \omega            \\
      Z_{L2} & = j L_2 \omega            \\
      Z_{C1} & = -j \frac{1}{C_1 \omega} \\
      Z_{C2} & = -j \frac{1}{C_2 \omega}
    \end{aligned}
  \end{equation*}
\end{figure}

\subsection{Výpočet proudů}
\begin{figure}[H]
  Sestavené rovnice smyček podle II. Kirchhoffova zákona

  \begin{equation*}
    \begin{aligned}
      -U_1 + Z_{L1} I_A + R_1 (I_A - I_B) + Z_{C2} (I_A + I_C) & = 0 \\
      -U_2 + R_1 (I_B - I_A) + Z_{C1} I_B                      & = 0 \\
      -U_2 + Z_{C2} (I_C + I_A) + Z_{L2} I_C + R_2 I_C         & = 0
    \end{aligned}
  \end{equation*}

  Po vytknutí proudů dostáváme:

  \begin{equation*}
    \begin{aligned}
      I_A (Z_{L1} + R_1 + Z_{C2}) + I_B (-R_1) + I_C (Z_{C2}) & = U_1 \\
      I_A (-R_1) + I_B (R_1 + Z_{C1})                         & = U_2 \\
      I_A (Z_{C2}) + I_C (Z_{C2} + Z_{L2} + R_2)              & = U_2
    \end{aligned}
  \end{equation*}


  Následně soustavu rovnic převedeme do matice

  $$
    \begin{pmatrix}
      Z_{L1}+R_1+Z_{C2} & -R_1       & Z_{C2}            \\
      -R_1              & R_1+Z_{C1} & 0                 \\
      Z_{C2}            & 0          & Z_{C2}+Z_{L2}+R_2
    \end{pmatrix}
    \cdot
    \begin{pmatrix}
      I_A \\
      I_B \\
      I_C
    \end{pmatrix}
    =
    \begin{pmatrix}
      U_1 \\
      U_2 \\
      U_2
    \end{pmatrix}
  $$
\end{figure}

\subsection{Výpočet napětí a fázového posunu}
\begin{figure}[H]
  Po vypočítání matice už snadno vypočítáme napětí $|U_{L2}|$ a fázový posun \varphi_{L2}

  \begin{equation*}
    \begin{aligned}
      I_{L2}       & = I_C                                                                               \\
      U_{L2}       & = Z_{L2} I_{L2}                                                                     \\
      |U_{L2}|     & = |Z_{L2} I_{L2}|                                                                   \\
      \varphi_{L2} & = \mathrm{tan^{-1}}\left(\frac{U_{L2imag}}{U_{L2real}}\right) \cdot \frac{180}{\pi}
    \end{aligned}
  \end{equation*}
\end{figure}

\subsection{Dosazení}
\begin{figure}[H]

  $$
    \begin{pmatrix}
      10 + 78.7071j & -10         & -24.9655j    \\
      -10           & 10 -9.2264j & 0            \\
      -24.9655j     & 0           & 13 + 8.0212j
    \end{pmatrix}
    \cdot
    \begin{pmatrix}
      I_A \\
      I_B \\
      I_C
    \end{pmatrix}
    =
    \begin{pmatrix}
      3 \\
      4 \\
      4
    \end{pmatrix}
  $$

  $$
    \begin{pmatrix}
      I_A \\
      I_B \\
      I_C
    \end{pmatrix}
    =
    \begin{pmatrix}
      0.1712 - 0.0356j \\
      0.3263 + 0.2655j \\
      0.4193 + 0.0702j
    \end{pmatrix}
  $$

  \begin{equation*}
    \begin{aligned}
      I_{L2}                                                                      & = (0.4193 + 0.0702j) A                                       \\
      U_{L2}                                                                      & = 32.9867j \cdot (0.4193 + 0.0702j) = (-2.3142 + 13.8305j) V \\
      |U_{L2}|                                                                    & = \underline{\underline{14.0228 V}}                          \\
      \mathrm{tan^{-1}}\left(\frac{13.8305}{-2.3142}\right) \cdot \frac{180}{\pi} & = -80.501 \degree
    \end{aligned}
  \end{equation*}

  Úhel fázového posunu vyšel ve špatném kvadrantu, proto je potřeba přičíst $180 \degree$

  $$ \varphi_{L2} = \mathrm{tan^{-1}}\left(\frac{13.8305}{-2.3142}\right) \cdot \frac{180}{\pi} + 180 \degree = \underline{\underline{ 99.499 \degree }}$$

\end{figure}
