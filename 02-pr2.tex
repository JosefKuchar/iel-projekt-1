\section{Příklad 2}
% Jako parametr zadejte skupinu (A-H)
\druhyZadani{C}

\subsection{Výpočet $R_i$}
\begin{figure}[H]
  V obvodu vyzkratujeme zdroj a odstraníme rezistor $R_1$. Následně spočítáme odpor mezi svorkami, kde byl resitor $R_1$ původně.

  \begin{circuitikz}
    \draw
    (0,0) -- (0,3)
    (0,3) to[R, l=$R_2$] (3,3)
    (3,3) to[R, l=$R_4$] (6,3)
    (3,3) to[R, l=$R_3$] (3,0)
    (3,0) to[R, l=$R_5$] (6,0)
    (6,0) -- (6,3)
    (0,0) -- (3,0)
    (0,3) node[ocirc]{} (0,3)
    (3,3) node[ocirc]{} (3,3)
    {[anchor=south] (0,3) node{A} (3,3) node{B}}
    ;
  \end{circuitikz}

  \begin{equation*}
    \begin{aligned}
      R_{45}  & = R_4 + R_5                               \\
      R_{345} & = \frac{R_3 \cdot R_{45}}{R_3 + R_{45}}   \\
      R_i     & = \frac{R_2 \cdot R_{345}}{R_2 + R_{345}}
    \end{aligned}
  \end{equation*}
\end{figure}

\subsection{Výpočet $U_i$}
\begin{figure}[H]
  Z původního obvodu odstraníme rezistor $R_2$. Poté spočítáme $R_{ekv}$, pro zjednodušení můžeme použít $R_{345}$ z předchozího výpočtu.
  Po spočítání $R_{ekv}$ můžeme spočítat proud $I_x$ a následně $U_i$, protože $U_i = U_{R2}$.

  \begin{circuitikz}
    \draw
    (0,0) to[dcvsource, v<=$U$, i=$I_x$](0,3)
    (0,3) to[R, l=$R_2$, v=$U_i$] (3,3)
    (3,3) to[R, l=$R_4$] (6,3)
    (3,3) to[R, l=$R_3$] (3,0)
    (3,0) to[R, l=$R_5$] (6,0)
    (6,0) -- (6,3)
    (0,0) -- (3,0)
    (0,3) node[ocirc]{} (0,3)
    (3,3) node[ocirc]{} (3,3)
    {[anchor=south] (0,3) node{A} (3,3) node{B}}
    ;
  \end{circuitikz}
  \begin{equation*}
    \begin{aligned}
      R_{ekv} & = R_{345} + R_2     \\
      I_x     & = \frac{U}{R_{ekv}} \\
      U_i     & = R_2 \cdot I_x
    \end{aligned}
  \end{equation*}
\end{figure}

\subsection{Výpočet proudu a napětí na $R_1$}
\begin{figure}[H]
  Už máme $U_i$ a $R_i$, takže můžeme podle Ohmova zákona snadno spočítat $I_{R1}$ a $U_{R1}$.
  \begin{circuitikz}
    \draw
    (0,0) to[dcvsource, v<=$U_i$](0,3)
    (0,3) to[R, l=$R_i$] (3,3)
    (3,3) to[R, l=$R_1$, i=$I_{R1}$, v=$U_{R1}$] (3,0)
    (0,0) -- (3,0)
    ;
  \end{circuitikz}
  \begin{equation*}
    \begin{aligned}
      I_{R1} & = \frac{U_i}{R_i + R_1} \\
      U_{R1} & = R_1 \cdot I_{R1}
    \end{aligned}
  \end{equation*}
\end{figure}

\subsection{Dosazení}
\begin{figure}[H]
  \begin{equation*}
    \begin{aligned}
      R_{45}  & = R_4 + R_5 = 240 + 450 = 690 \ohm                                                                                  \\
      R_{345} & = \frac{R_3 \cdot R_{45}}{R_3 + R_{45}} = \frac{630 \cdot 690}{630 + 690} = \frac{7245}{22} \ohm                    \\
      R_i     & = \frac{R_2 \cdot R_{345}}{R_2 + R_{345}} = \frac{220 \cdot \frac{7245}{22}}{220 + \frac{7245}{22}} = 131.8908 \ohm \\
      R_{ekv} & = R_{345} + R_2 = \frac{7245}{22} + 220 = 549.3182 \ohm                                                             \\
      I_x     & = \frac{U}{R_{ekv}} = \frac{200}{549.3182} = 0.3641 A                                                               \\
      U_i     & = R_2 \cdot I_x = 220 \cdot 0.3641 = 80.0993 V                                                                      \\
      I_{R1}  & = \frac{U_i}{R_i + R_1} = \frac{80.0993}{131.8908 + 70} = \underline{\underline{3.9675 \cdot 10^{-1} A}}            \\
      U_{R1}  & = R_1 \cdot I_{R1} = 70 \cdot 3.9675 \cdot 10^{-1} = \underline{\underline{27.7722 V}}
    \end{aligned}
  \end{equation*}
\end{figure}
